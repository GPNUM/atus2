\documentclass[12pt]{article}
\usepackage[left=2cm,right=2cm,top=20mm,bottom=2cm]{geometry}\textheight=235mm
\usepackage[utf8]{inputenc}
\usepackage[english]{babel}
\usepackage{amsmath,amsfonts,amssymb,amsthm}
\usepackage{graphicx}
\usepackage{braket}
\usepackage{listings}
\usepackage{bbm}
\usepackage{cite}
\usepackage{minted}
\usepackage{color}
\usepackage[T1]{fontenc}
\usepackage[usenames,dvipsnames,svgnames,table]{xcolor}
\usepackage{tikzstyle}
\usepackage{tocloft}
\usepackage{tcolorbox}
\newcommand{\ic}{\mathbbm{i}}
\newcommand{\ubar}[1]{\underaccent{\bar}{#1}}

\begin{document}


\section{Theory}
In this tutorial we will focus on the numerical implementation of a matter wave interferometer in 
the Mach-Zehnder configuration. As beam splitters and mirrors are used to manipulate quantum gases, we will focus on the physics of the beam splitting process (the mirror is just a special case of a beam splitter and therefore automatically included in the following descriptions).
The beam splitter is approximated as a three-level system. 
\begin{figure}[h]
\centering
\begin{tikzpicture}[node distance=1cm, every edge/.style={link},every node/.style={align=center}]

  \node [] (gs1) at (0cm,0cm) {};
  \node [] (gs1_r) at ($ (gs1.east) + (1.0,0.0) $) {};
  \path[-,cap=round,draw=black,line width=2.0] (gs1) -- (gs1_r);
  \node [] (gs2) at ($ (gs1.east) + (2.0,2.0) $) {};
  \node [] (gs2_r) at ($(gs2.east) + (1.0,0.0)$) {};
  \path[-,cap=round,draw=black,line width=2.0] (gs2) -- (gs2_r);
  \node [] (ex1) at (1cm,5cm) {};
  \node [] (ex1_r) at ($(ex1.east) + (1.0,0.0)$) {};
  \path[-,cap=round,draw=black,line width=2.0] (ex1) -- (ex1_r);
  \node [] (virt1) at ($(ex1.south) - (0.0,0.8)$) {};
  \node [] (virt1_r) at ($(virt1.east) + (1.0,0.0)$) {};
  \path[dashed,cap=round,draw=black,line width=2.0] (virt1) -- (virt1_r);
 \path[photon,->] ($(gs1.north)+(0.5,0)$) -- ($(virt1.south)+(0.5,0)$);
  \path[photon,->] ($(virt1.south)+(0.5,0)$) -- ($(gs2.north)+(0.5,0)$);
  \draw[decorate,decoration={brace,amplitude=6pt},xshift=-0pt,yshift=0pt] (ex1_r) -- (virt1_r) node [black,midway,xshift=0.6cm] {$\Delta $};   
  \node [] at ($(gs1)-(0.5,0.0)$) {$\ket{1}$}; 
  \node [] at ($(gs2)+(1.5,0.0)$) {$\ket{3}$}; 
  \node [] at ($(ex1)-(0.5,0.0)$) {$\ket{2}$}; 
  \node [] (lab_om2) at ($(gs2)+(1.0,1.0)$) {$\omega_2,k_2$}; 
  \node [] (freq1) at ($(lab_om2)-(2.8,1.0)$) {$\omega_1,k_1$}; 
%  %%\node [left=of freq1,yshift=2cm] {{\bf Bragg diffraction}};
%  %\node (sup) [left=of freq1,yshift=-2cm,align=left] {$E_0 \cos(k_1 x -\omega_1 t) +$ \\$E_0 \cos(-k_2 x-\omega_2 t )$};
%  %\node [right=of sup,align=left] { Dipole coupling: $\hat{d} \cdot \vec{E}$};
%   
\end{tikzpicture}
%
\label{fig:3ns}
\end{figure}

The states $\ket{1}$ and $\ket{2}$ having energies $E_1$ and $E_2$ are coupled by means of the first laser with angular frequency 
$\omega_1$ and wave vector $k_1$. A second laser ($\omega_2,\, k_2$) induces the transition between the states $\ket{2}$ and $\ket{3}$. Here, $\Delta$ is the detuning of the laser frequency w.r.t. the energy of the state $\ket{2}$. This leads to a supression of the occupation of state $\ket{2}$.

\subsection{Single diffraction: Bragg beam splitter part I}
\paragraph{Schrödinger equation}
Now we will focus on Bragg beam splitters which can be approximated as a two-level system with state vectors $\ket{1}$ und $\ket{2}$, which are orthogonal to each other, i.e. $\langle 1 | 2 \rangle=0$. 
The dynamics of the quantum system including the interaction with the external beam-splitter pulses is given by the Schrödinger equation
\begin{align}
  \ic\hbar \partial_t \Psi = -\frac{\hbar^2}{2m} \nabla^2\Psi + V\Psi, \label{eq:sgl_ini}
\end{align}
where $V=V_{ext}+V_L$. $V_{ext}$ is an arbitrary external potential, whereas $V_L$ is the interaction potential due to the interaction of the lasers with the quantum system.
Both state vectors of the two-level system form a basis, therefore the wavefunction can be expressed as the linear combination 
\begin{align}
  \ket{\Psi} = 
  \begin{pmatrix}
    \Psi_1 \\
    \Psi_2 
  \end{pmatrix} =
  c_1\ket{1}+ c_2\ket{2}, 
\end{align}
where  the components are functions of space and time, e.g.  $\Psi_1(\vec{r},t)$.
Two counter propagating laser beams ($\omega_1,\, k_1$) and ($\omega_2,\, -k_2$) lead to a coupling of the ground state $\ket{1}$ with the excited state $\ket{2}$. We assume the electric field $\vec{E}$ as a plane wave, i.e. one can set two components to zero and choose $(E,0,0)$, then the dipole coupling reads $\vec{d} \cdot \vec{E} = d\cdot E$, where $\vec{d}$ is the dipole moment. 
The Schrödinger equation now reads
\begin{tcolorbox}
\begin{align}
  \ic \hbar \partial_t \begin{pmatrix}
    \Psi_1 \\
    \Psi_2 
  \end{pmatrix} = 
  -\frac{\hbar^2}{2m}\nabla^2\begin{pmatrix}
    \Psi_1 \\
    \Psi_2 
  \end{pmatrix} +
  \begin{pmatrix}
    0 & dE(\vec{r},t) \\
    dE^*(\vec{r},t) & E_2 
  \end{pmatrix}\begin{pmatrix}
    \Psi_1 \\
    \Psi_2 
  \end{pmatrix}.
\end{align}
\end{tcolorbox}
Note that the ground state energy has been set to $E_1=0$. 

\paragraph{Gross-Pitaevskii equation}
As for the case of the Gross-Pitaevskii equation we have
\begin{eqnarray}
i \hbar \partial_t \Psi = -\frac{\hbar^2}{2m}\nabla^2\Psi+V(\vec{r})\Psi+g|\Psi|^2\Psi,\label{eq:GPE_dyn}
\end{eqnarray}
where $g$ is given by the s-wave scattering length $\frac{4\pi \hbar^2}{m}a_s$ and the order parameter ("wavefunction of the condensate") is a function of space and time, i.e.  $\Psi(\vec{r},t)$. Again, the wavefunction is decomposed by means of the state vectors of the two-level system  leading to a matrix-vector Gross-Pitaevskii equation with a $2 \times 2$ Hamiltonian $H_{ij}$
\begin{align}
  \ic \hbar \partial_t \begin{pmatrix}
    \Psi_1 \\
    \Psi_2 
  \end{pmatrix} = 
  \begin{pmatrix}
    H_{11}& H_{12} \\
    H_{21} & H_{22}
  \end{pmatrix}\begin{pmatrix}
    \Psi_1 \\
    \Psi_2 
  \end{pmatrix}.
\end{align}
The diagonal elements of the Hamiltonian $H_{ij}$ read

\begin{eqnarray}
H_{11}& = &  -\frac{\hbar^2}{2m}\nabla^2 +g_{11} |\Psi_1|^2 +g_{12} |\Psi_2|^2 \\
H_{22}& = &  -\frac{\hbar^2}{2m}\nabla^2 +\hbar \omega_{12}+g_{21} |\Psi_1|^2 +g_{22} |\Psi_2|^2 ,
\end{eqnarray}
and the non-diagonal elements describe the interaction with the lasers
\begin{eqnarray}
V_{12} =V_{21}^* = dE(\vec{r},t).
\end{eqnarray}

Note that $g_{12}=g_{21}$ and the nonlinearity factor is determined by the s-wave scattering length $a_{ij}$, that is $g_{ij}=\frac{4\pi \hbar^2}{m}a_{ij}$.
Note that the the two states $\ket{1,2}$ interact mutually with each other. This is encoded in the $(ij)$-components of the nonlinearity factor $g_{ij}$ what is similar to the case of two-species BECs. Therefore, the Hamilton above is capapble of describing this case, too. 
\subsection{Single diffraction: Bragg beam splitter part II}
From now on we will focus on the Gross-Pitaevskii equation as the linear case is recovered by just setting the nonlinear term to zero.  
In order to specify the interaction with the laser field, we describe the electrical field by a superposition of two counter propagating electric fields
\begin{align}
  E &= A\left[\cos(k_1x-\omega_1 t-\phi_L(x)/2)+\cos(-k_2x-\omega_2 t+\phi_L(x)/2)\right] ,
  \label{eq:efeld}
\end{align}
where $\phi_L(x)$ is a laser phase which can be spatially dependent. This term is important when one wishes to include phase front errors of the lasers. 
Applying the Rotating-Wave Approximation neglects fastly oscillating terms, i.e.
\begin{align}
  \cos(\omega t) &= \frac{1}{2}\left( e^{\ic\omega t}+e^{-\ic\omega t}\right) 
                = \frac{1}{2}e^{\ic\omega t}\left( 1+ e^{-2\ic\omega t}\right) 
                \approx \frac{1}{2}e^{\ic\omega t}.
\end{align}
The magnitude of $\omega$ in real experiments is  usually of the order THz.

The electric field now reads
\begin{align}
  E(x,t) &= A\left[e^{\ic (k_1x-\omega_1 t-\phi_L(x)/2)}+e^{\ic(-k_2x-\omega_2 t+
    \phi_L(x)/2)} \right] 
\end{align}
Now the GPE is transformed to the Dirac picture (interaction picture) by means of the unitary matrix $U=\exp{(i H_0 t/\hbar)}$. For the wavefunction this yields $\tilde{\Psi} = U^\dagger \Psi$ and the time-independent operator $H_0$ is given by
\begin{align}
  H_0 = \begin{pmatrix}
    0 & 0 \\
    0 & \frac{\hbar(\omega_1+\omega_2)}{2} \\
  \end{pmatrix}.
\end{align}
The matrix elements describing the interaction with the lasers in the dipole approximation are given by the non-diagonal elements
\begin{align}
   V_{12}=V_{21}^*= d_{12}\frac{A}{2}\left[e^{\ic\left(k_1x-\Delta \omega t - 
  \varphi_L/2 \right)}+ e^{-\ic\left(k_2x-\Delta\omega t - \varphi_L/2 
  \right)}\right]\label{eq:vlicht},
\end{align}
where we defined $\Delta\omega := \frac{\omega_1-\omega_2}{2}$. The transformed interaction operator yields 
\begin{align}
  V_L = \begin{pmatrix}
    0 & V_{12}\\
    V_{12}^* & \Delta_1 
  \end{pmatrix},
\end{align}
where the detuning is given by 
$\Delta_1 := E_2-\frac{\hbar(\omega_1+\omega_2)}{2} $.
Note that the dependence on $\omega$ disappeared and instead the difference $\Delta \omega$ appears. In typical experiments the magnitudes are  $\omega \approx$ THz, whereas $\Delta \omega \approx$ kHz. This is quite practical for numerical similations in order to avoid  the introduction of several simultaneous time scales differing by orders of magnitude. 
The full Gross-Pitaevskii equation now reads
\begin{tcolorbox}
\begin{align}
  \ic \hbar \partial_t \begin{pmatrix}
    \Psi_1 \\
    \Psi_2 
  \end{pmatrix} =  -\frac{\hbar^2}{2m}\nabla^2
  \begin{pmatrix}
    \Psi_1 \\
    \Psi_2 
  \end{pmatrix}
  +  \begin{pmatrix}
    0 & V_{12}\\
    V_{12}^* & \Delta_1 
  \end{pmatrix}
  \begin{pmatrix}
    \Psi_1 \\
    \Psi_2 
  \end{pmatrix}+
\begin{pmatrix}
    g_{11}& g_{12}\\
    g_{12} & g_{22} 
  \end{pmatrix}   
   \begin{pmatrix}
    |\Psi_1|^2 \\
    |\Psi_2|^2 
  \end{pmatrix} 
  \begin{pmatrix}
    \Psi_1 \\
    \Psi_2 
  \end{pmatrix}. \label{eq:GPE_dyn_matrix}
\end{align}
\end{tcolorbox}
\textbf{Note: From now on $\Psi(\vec{r},t)$ is understood as a vector until otherwise stated!}
\subsection{Numerical solution of the time dependent GPE}
The dynamics of the quantum system is described by the equation (\ref{eq:GPE_dyn_matrix}) and the solution is formally given by the time evolution operator
\begin{eqnarray}
  \Psi(\vec{r},t) = \hat{T} \exp\left(-\frac{\ic}{\hbar}\int_{t_0}^t H(\tau)\,d\tau\right)
\Psi(\vec{r},t_0) \label{eq:Loesung},
\end{eqnarray}
where $\Psi(\vec{r},t_0)$ is the solution to initial time $t_0$ and
$\hat{T}$ is the time ordering operator. 

The Hamiltonian is split into the kinetic $\hat{K}$ and potential term $\hat{V}$ 
\begin{eqnarray}
\hat{H}=\hat{K}+\hat{V}.
\end{eqnarray}
Applying the operator splitting 
\begin{eqnarray}
  e^{\hat{K}+\hat{V}}\approx e^{\frac{1}{2}\hat{K}}e^{\hat{V}}e^{\frac{1}{2}\hat{K}}
\end{eqnarray}
the time propagated wavefunction $\Psi(\vec{r},t_0) \rightarrow \Psi(\vec{r},t_0+\Delta t)$ reads
\begin{tcolorbox}
\begin{eqnarray}
\Psi(\vec{r},t_0+\Delta t) \approx \exp\left(-\frac{\ic}{\hbar}\frac{\Delta t}{2}\hat{K}\right)
\exp\left(-\frac{\ic}{\hbar}\Delta t\hat{V}\right)\exp\left(-\frac{\ic}{\hbar}\frac{\Delta t}{2}\hat{K}\right)\Psi(\vec{r},t_0).
\end{eqnarray}
\end{tcolorbox}
Due to  the fact that the operators $\hat{K}$ and $\hat{V}$ do not commute, one needs to estimate the error. 
Utilizing the Baker-Campbell-Hausdorff equation
\begin{eqnarray}
e^{\hat{K}}e^ {\hat{V}}=e^{\hat{Z}},
\end{eqnarray} 
where $\hat{Z}$
\begin{eqnarray}
  \hat{Z}=\hat{K}+\hat{V}+\frac{1}{2}[\hat{K},\hat{V}]+\frac{1}{12}([\hat{K},
  [\hat{K},\hat{V}]]+[\hat{V},[\hat{V},\hat{K}]])+\ldots
\end{eqnarray}
one calculates the difference between the approximated solution and the real one, that is
\begin{eqnarray}
e^{\frac{\Delta t}{2} \hat{K}}e^{\Delta t \hat{V}}e^{\frac{\Delta t}{2} 
\hat{K}}-e^{\Delta t (\hat{K}+\hat{V})}, 
\end{eqnarray}
yielding an error of $\mathcal{O}(\Delta t^3)$.


\subsection{Kinetic Operator}
The kinetical part of the operator is described by 
$$\exp\left(-\frac{\ic}{\hbar}\frac{\Delta t}{2}\hat{K}\right) \Psi(r,t)$$.
It can be calculated numerically by means of pseudo-spectral methods, here by means of Fourier transformation. 
The Fourier transformation $\mathcal{F}$
transforms $\Psi$ to the momentum representaton $\Psi_p$.
For the kinetic operator this yields
$$-\hbar^2\nabla^2/2m\longrightarrow -\hbar^2k^2/2m.$$ 
By means of this the exponential kinetic operator function can be calculated easily and is applied to the wavefunction. 
After this, $\Psi_p$ is transformed back by the inverse Fourier transformation 
$\mathcal{F}^{-1}$ to position space. 
The complete solution scheme for the kinetic operator now reads:
\begin{eqnarray}
\exp\left(\ic\frac{\Delta t}{2}\frac{\hbar\nabla^2}{2m}\right)\Psi(r,t)=\mathcal{F}^{-1}
\left\{\exp\left(\ic\frac{\Delta t}{2}\frac{\hbar k^2}{2m}\right)\mathcal{F}\Psi(r,t)\right\}.
\end{eqnarray}

\subsection{Potential part}
The potential term is given by the light-matter interaction 
(\ref{eq:licht}) and the exponential function is 
\begin{align}
  \exp\left(-\frac{\ic}{\hbar}\Delta t \hat{V}\right) 
\end{align}
with the interaction    
\begin{eqnarray}
  \hat{V} = \begin{pmatrix}
  0 & V_{12} \\
V_{12}^* & \Delta_1.
\end{pmatrix}
\end{eqnarray} 
Because of the fact that $\hat{V}$ is hermitean, there always exists an unitary transformation $U$,
which converts $\hat{V}$ to diagonal form, i.e. $\hat{D}=U\hat{V}U^{\dagger}$. 
Writing the exponential function as a series expansion, 
it can be shown that
\begin{align}
  \exp\left(\hat{V}\right) = U^\dagger\exp\left(\hat{D}\right)U.
\end{align}
This can be utilized to calculate $\exp(\hat V)$. Therefore, the potential term yields
\begin{eqnarray}
\exp\left(-\frac{\ic\Delta t}{\hbar} \hat{V}\right) = U^\dagger 
\begin{pmatrix}
\exp(-\ic E_1\Delta t/\hbar) &  0\\ 0& \exp(-\ic E_2\Delta t/\hbar) 
\end{pmatrix} U \label{eq:num}
\end{eqnarray}
Here, $E_1$ and $E_2$ are the eigenvalues of $\hat{V}$.

\section{Dimensionless GPE}
\subsection{Time dependent Gross-Pitaevskii-Equation}
For 
\begin{align}
\Psi_{1,2} : \mathbb{R}^3 \rightarrow \mathbb{C} \nonumber
\end{align}
the coupled 2-species Gross-Pitaevskii equation reads
\begin{align}
i \hbar \partial_t \Psi_1 &= \left( -\dfrac{\hbar^2}{2m_1} \Delta + V_1 + g_{12} \vert\Psi_2\vert^2 + g_{11} \vert\Psi_1\vert^2\right)\Psi_1 ,\nonumber\\
i \hbar \partial_t \Psi_2 &= \left( -\dfrac{\hbar^2}{2m_2} \Delta + V_2 + g_{21} \vert\Psi_1\vert^2 + g_{22} \vert\Psi_2\vert^2\right)\Psi_2 \nonumber,
\end{align}
and leads to
\begin{align}
i\partial_t \Psi_1 &= \left(- \dfrac{\hbar}{2m_1} \Delta +\dfrac{1}{\hbar} V_1 + \dfrac{1}{\hbar} g_{12} \vert\Psi_2\vert^2 + \dfrac{1}{\hbar} g_{11} \vert\Psi_1\vert^2\right)\Psi_1 \label{tdnlse01},\\
i\partial_t \Psi_2 &= \left(- \dfrac{\hbar}{2m_2} \Delta + \dfrac{1}{\hbar} V_2 + \dfrac{1}{\hbar} g_{21} \vert\Psi_1\vert^2 + \dfrac{1}{\hbar} g_{22} \vert\Psi_2\vert^2\right)\Psi_2 \label{tdnlse02}.
\end{align}

\paragraph{Dimensionless equation}
We rescale with $L$ which is a length scale (e.g. typical extension of the BEC) and $T$, being a time scale (e.g. typical time connected to the harmonic trap frequency $\omega$) for $n$-dimensions according to
\begin{align}
t \rightarrow Tt, \mathbf{x} \rightarrow L \mathbf{x}, \Psi_1 \rightarrow \sqrt{N_1} \Psi_1 / L^{n/2}, \Psi_2 \rightarrow \sqrt{N_2} \Psi_2 / L^{n/2} \nonumber
\end{align}
and equations (\ref{tdnlse01}) and (\ref{tdnlse02}) become
\begin{align}
\dfrac{i}{T}\partial_t \Psi_1 &= \left( -\dfrac{\hbar}{2m_1L^2} \Delta + \dfrac{1}{\hbar} V_1(L\mathbf{x},Tt) + \dfrac{1}{\hbar} \dfrac{g_{12}N_2}{L^n} \vert\Psi_2\vert^2 + \dfrac{1}{\hbar} \dfrac{g_{11}N_1}{L^n} \vert\Psi_1\vert^2\right)\Psi_1 \nonumber \\
\dfrac{i}{T}\partial_t \Psi_2 &= \left( -\dfrac{\hbar}{2m_2L^2} \Delta + \dfrac{1}{\hbar} V_2(L\mathbf{x},Tt) + \dfrac{1}{\hbar} \dfrac{g_{21}N_1}{L^n} \vert\Psi_1\vert^2 + \dfrac{1}{\hbar} \dfrac{g_{22}N_2}{L^n} \vert\Psi_2\vert^2\right)\Psi_2, \nonumber
\end{align}
from which we get
\begin{center}
\begin{tcolorbox}
\begin{eqnarray}
i\partial_t \Psi_1 &= \left( -\tilde{\alpha}_1 \Delta + \dfrac{1}{\hbar} T V_1(L\mathbf{x},Tt) +  \gamma_{12} \vert\Psi_2\vert^2 + \gamma_{11} \vert\Psi_1\vert^2\right)\Psi_1 \label{tdnlse1} \\
i\partial_t \Psi_2 &= \left( -\tilde{\alpha}_2 \Delta + \dfrac{1}{\hbar} T V_2(L\mathbf{x},Tt) +  \gamma_{21} \vert\Psi_1\vert^2 + \gamma_{22} \vert\Psi_2\vert^2\right)\Psi_2, \label{tdnlse2}
\end{eqnarray}
\end{tcolorbox}
\end{center}
where we introduced as a scale factor $\tilde{\alpha}_X=\dfrac{\hbar T}{2 m_X L^2}$ and rewrote the nonlinearity parameter according to 
\begin{align}
\gamma_{11} &:= \frac{g_{11} T N_1}{\hbar L^n}=\dfrac{4\pi \hbar}{m_1} \dfrac{TN_1}{L^n} a_{11} \\
\gamma_{12} &:= \frac{g_{12} T N_2}{\hbar L^n}=\dfrac{2\pi \hbar (m_1+m_2)}{m_1m_2} \dfrac{TN_2}{L^n} a_{12} \\
\gamma_{21} &:= \frac{g_{21} T N_1}{\hbar L^n}=\dfrac{2\pi \hbar (m_1+m_2)}{m_1m_2} \dfrac{TN_1}{L^n} a_{21} \\
\gamma_{22} &:= \frac{g_{22} T N_2}{\hbar L^n}=\dfrac{4\pi \hbar}{m_2} \dfrac{TN_2}{L^n} a_{22},
\end{align}
using
\begin{eqnarray}
 g_{ii}= 4 \pi \hbar^2 \dfrac{a_{ii}}{m_i} \qquad g_{ij}= 2 \pi \hbar^2 a_{ij}\dfrac{m_i+m_j}{m_i m_j}.
\end{eqnarray}
Note the simultaneous $T,L$ dependency of the dimensionless parameters $\gamma_{ij}$ !  

\subsection{Time independent Gross-Pitaevskii equation}
\paragraph{Dimensionless equation}
One obtains the time-independent dimensionless equation of motion by making the following ansatz for the order parameter
\begin{equation}
 \Psi_j(\mathbf{r},t):=\Psi_j(\mathbf{r}) \exp{\left(-i/\hbar \mu_j t\right)},
\end{equation}
where $\mu$ is the chemical potential.  
Setting $\Psi_i:=\Psi_i(\mathbf{r})$ and applying the ansatz in equations (\ref{tdnlse1}) and (\ref{tdnlse2}) one gets
\begin{eqnarray}
-i\frac{\mu_1}{\hbar}\Psi_1&=&\left(-\tilde{\alpha}_1 \Delta + \dfrac{1}{\hbar} TV_1(L\mathbf{x}) + \gamma_{12} \vert\Psi_2\vert^2 + \gamma_{11} \vert\Psi_1\vert^2\right)\Psi_1, \nonumber \\
-i\frac{\mu_2}{\hbar}\Psi_2&=&\left(-\tilde{\alpha}_2 \Delta + \dfrac{1}{\hbar} TV_2(L\mathbf{x}) + \gamma_{21} \vert\Psi_1\vert^2 + \gamma_{22}\vert\Psi_2\vert^2\right)\Psi_2. \nonumber
\end{eqnarray}
We redefine the chemical potential by means of $\varepsilon_i:=-\dfrac{ \mu_i}{\hbar}$ and get the corresponding time-independent equations to equations (\ref{tdnlse1}) and (\ref{tdnlse2}) ,
\begin{center}
\begin{tcolorbox}
\begin{eqnarray}
\varepsilon_1 \Psi_1 &=&\left(-\tilde{\alpha}_1 \Delta + \dfrac{1}{\hbar} V_1(L\mathbf{x}) + \gamma_{12} \vert\Psi_2\vert^2 + \gamma_{11}\vert\Psi_1\vert^2\right)\Psi_1, \label{tinlse1} \\
\varepsilon_2 \Psi_2 &=&\left(-\tilde{\alpha}_2 \Delta + \dfrac{1}{\hbar} V_2(L\mathbf{x}) + \gamma_{21} \vert\Psi_1\vert^2 + \gamma_{22} \vert\Psi_2\vert^2\right)\Psi_2. \label{tinlse2}
\end{eqnarray}
\end{tcolorbox}
\end{center}


\subsection{External potentials }
\paragraph{Dipole potential (trapping potential)}
\begin{align}
V_X = - \dfrac{P Re(\alpha_X)}{\pi c \varepsilon_0} \dfrac{1}{w^2(z)} \exp\left( -2\dfrac{|\mathbf{r}|^2}{w^2(z)} \right) \label{optdippot1},
\end{align}
where index $X=\{Rb, K\}$ and $|\mathbf{r}|^2= x^2+y^2+z^2 := \rho^2+z^2$.
\begin{align}
w^2(z) = w_0^2\left(1+z^2/z^2_R\right)
\end{align}

\begin{align}
z_R = \dfrac{\pi w_0^2}{\lambda M^2} \,\,\text{ Rayleigh length} 
\end{align}
\paragraph{Approximated potential}
\begin{eqnarray}
V_X &=& -\hat{V}_X\left(1-2\left(\frac{\rho}{w_0}\right)^2-\left(\frac{z}{z_R}\right)^2\right)\\
&:=& \frac{1}{2}m_X\left(\omega_{\rho}^2 \rho^2+\omega_z^2 z^2\right)-\hat{V}_X \label{optdippot2},
\end{eqnarray}
where 
\begin{eqnarray}
\hat{V}_X&:=&\dfrac{P Re(\alpha_X)}{\pi c \varepsilon_0 w_0^2}\\
\omega_{\rho}^2&:=&\frac{4\hat{V}_X}{m_X w_0^2} \quad \omega_z^2:= \frac{2\hat{V}_X}{m_X z_R^2} .
\end{eqnarray}
\paragraph{Dimensionless dipole potential}
The argument and the function of the dipole potential are scaled according to
\begin{align}
(\mathbf{r}) \rightarrow (L\mathbf{r}), \qquad V_X(\mathbf{r}) \rightarrow \dfrac{T}{\hbar} V_X(L\mathbf{r}).
\end{align}
This yields for (\ref{optdippot1})
\begin{eqnarray}
 V_X &=& - \dfrac{P Re(\alpha_X)}{\pi c \varepsilon_0} \dfrac{1}{w^2(z)} \exp\left( -2\dfrac{|\mathbf{r}|^2}{w^2(z)} \right) \\
 &\rightarrow& - \dfrac{P Re(\alpha_X) T}{\pi \hbar c \varepsilon_0 w_0^2\left(1+(Lz)^2/z^2_R\right)}\exp\left( -\dfrac{2L^2|\mathbf{r}|^2}{w_0^2\left(1+(Lz)^2/z^2_R\right)} \right),
\end{eqnarray}
thus
\begin{center}
\begin{tcolorbox}
\begin{eqnarray}
 V_X(\mathbf{r})=\dfrac{\eta_X}{1+\kappa^2} \exp{\left(-\dfrac{\delta |\mathbf{r}|^2}{1+\kappa^2}\right)}. 
\end{eqnarray}
\end{tcolorbox}
\end{center}
Here is $\eta_X =  \dfrac{P Re(\alpha_X) T}{\pi\hbar c \varepsilon_0 w_0^2}, \kappa = L / z_R $ and $\delta = 2 L^2 / w_0^2$.
\paragraph{Dimensionless Newtonian gravity potential}
Again we have 
\begin{align}
(\mathbf{r}) \rightarrow (L\mathbf{r}), \qquad V_X(\mathbf{r}) \rightarrow \dfrac{T}{\hbar} V_X(L\mathbf{r}).
\end{align}
and for the Newtonian potential we write $V(z)_X= m_X g z$.
This transforms to 
\begin{center}
\begin{tcolorbox}
\begin{eqnarray}
m_Xgz\rightarrow \dfrac{T L}{\hbar} m_X g z :=\beta_X z,
\end{eqnarray}
\end{tcolorbox}
\end{center}
with $\beta_X=\dfrac{T L}{\hbar} m_X$.
\paragraph{Dimensionless harmonic trap potential}
Shall $V_X(\mathbf{r})$ be a time independent harmonic trap potential which in the general case reads
\begin{eqnarray}
 V_X(\mathbf{r})=\frac{1}{2} m_X \left(\omega^2_x x^2+\omega^2_y y^2+\omega^2_z z^2\right),
\end{eqnarray}
then the rescaled and dimensionless equation reads
\begin{center}
\begin{tcolorbox}
\begin{eqnarray}
V_X(\mathbf{r}) &=& \dfrac{1}{2}\dfrac{TL^2}{\hbar} m_X \left(\omega^2_x x^2+\omega^2_y y^2+\omega^2_z z^2\right)\\
&:=& \tilde{\omega}^2_{X,x} x^2+\tilde{\omega}^2_{X,y} y^2+\tilde{\omega}^2_{X,z} z^2,
\end{eqnarray}
\end{tcolorbox}
\end{center}
with $\tilde{\omega}^2_{X,i}:=\dfrac{1}{2}\dfrac{TL^2}{\hbar} m_X \omega^2_i$.
\paragraph{Dimensionless approximated optical dipole potential}
For the approximated dipole potential (\ref{optdippot2}) the dimensionless equation reads
\begin{eqnarray}
V_X(\mathbf{r}) & = & -\dfrac{\hat{V}_X T}{\hbar}\left(1-L^2\left(2\left(\frac{\rho}{w_0}\right)^2-\left(\frac{z}{z_R}\right)^2\right)\right)\\
& = & \frac{1}{2}\dfrac{TL^2}{\hbar}m_X\left(\omega_{\rho}^2 \rho^2+\omega_z^2 z^2\right)-\dfrac{T}{\hbar}\hat{V}_X, 
\end{eqnarray}
thus,
\begin{center}
\begin{tcolorbox}
\begin{eqnarray}
V_X(\mathbf{r}) =  \tilde{\omega}_{X,\rho}^2 \rho^2+\tilde{\omega}_{X,z}^2 z^2-\dfrac{T}{\hbar}\hat{V}_X.  
\end{eqnarray}
\end{tcolorbox}
\end{center}


\subsection{List of all parameters} \label{Parameterlist}
 \begin{align}
 \epsilon_1 &:= -\dfrac{T \mu_1}{\hbar} \\
 \epsilon_2 &:= -\dfrac{T \mu_2}{\hbar} \\
 \kappa &:= L / z_R \\
 \delta &:= 2 L^2 / w_0^2 \\
 \eta_1 &:= \dfrac{P Re(\alpha_1) T}{\pi\hbar c \varepsilon_0 w_0^2}\\
 \eta_2 &:= \dfrac{P Re(\alpha_2) T}{\pi\hbar c \varepsilon_0 w_0^2}\\
 \tilde{\alpha}_1 &:=\dfrac{\hbar T}{2 m_1 L^2}\\
 \tilde{\alpha}_2 &:=\dfrac{\hbar T}{2 m_2 L^2}\\
 \beta_1 &:= \dfrac{T L m_1 g_F }{\hbar} \\
 \beta_2 &:=  \dfrac{T L m_2 g_F }{\hbar} \\
 \gamma_{11} &:= \frac{g_{11} T N_1}{\hbar L^n}\\
 \gamma_{12} &:= \frac{g_{12} T N_2}{\hbar L^n}\\
 \gamma_{21} &:= \frac{g_{21} T N_1}{\hbar L^n} \\
 \gamma_{22} &:= \frac{g_{22} T N_2}{\hbar L^n}\\
 \tilde{\omega}^2_{X,i}&:=\dfrac{1}{2}\dfrac{TL^2}{\hbar} m_X \omega^2_i
 \end{align}

\section{Example}
\label{sec:example}
In the following we present a brief introduction to our simulation package by discussing
a Bragg beam splitter as it is used in atom interferometry. 
The files used in this example are located in
the directory \mintinline{bash}{atus2/xml/bragg}.\\
To calculate the time evolution of the wave packet 
we need, according to equation XYZ, the initial wave function $\Psi_0(t=0)$ of the
quantum system.
These wave functions are generated by the
\mintinline{bash}{gen_psi_0} tool which is an abbreviation for "generate $\Psi_0$".
$\Psi_0$ is composed of two wave functions; the ground state $\Psi_1$
and the excited state $\Psi_2$. The wave functions are generated by
\begin{minted}{bash}
$ gen_psi_0 gauss.xml
$ gen_psi_0 zero.xml
\end{minted}
respectively. An example for a time evolution of a wave packet is given by the file 
\mintinline{bash}{bragg_ad.xml}. The time evolution is divided into a beam splitter
and free propagation phase. In case of Bragg beam splitters,
interferometer sequences are started by the command:
\begin{minted}{bash}
$ bragg bragg_ad.xml
\end{minted}
%A more in depth introduction for possible parameters is found in the following chapter.
Wave functions are written to file in binary format. Files can be converted to plain
text files with the gpo3 tool:
\begin{minted}{xml}
$ gpo3 example.bin > example.txt
\end{minted}
These files can be plotted with the usual programs like gnuplot, phython, matlab, ...
For instance in gnuplot, plotting the final ground state $|\Psi_1|^2$ (after a propagation time $t=7100$) is done by the following commands:
\begin{minted}{gnuplot}
  p "<gpo3 7100.000_1.bin" w l
\end{minted}
The excited state $|\Psi_2|^2$ is plotted by 
\begin{minted}{gnuplot}
  p "<gpo3 7100.000_2.bin" w l
\end{minted}
In this case no intermediate file, i.e. \mintinline{bash}{example.txt}, is created. 
For two or three dimensional wave functions slices in only one direction can be plotted.  
Additionally the real or imaginary part can be plotted. Example:
\begin{minted}{gnuplot}
 p "<gpo3 -i 8 --re 2d_func.bin" w l
\end{minted}
Here the x coordinate is fixed at point 8. Use \mintinline{bash}{gpo3 --help} to see
all possible options.

\section{Introduction to our xml-files}
\label{sec:introduction_to_xml_files}
In this section we discuss the structure, options and tags of our xml-files.

\subsection{Generating initial wave functions}
\label{sub:generating_initial_wave_functions}

In general, our xml tags are encased by 
\mintinline{xml}{<SIMULATION>...</SIMULATION>}.
Below is a minimal, annotated example file for generating an initial wave function
(see \mintinline{xml}{atus2/xml/bragg/gauss.xml}).
\inputminted{xml}{gauss.xml} \noindent 
It is important to note that the number of grid points \mintinline{xml}{<NX>} 
should be a multiple of 2, because this leads to faster FFTs.
The shape of the wave function is set with the \mintinline{xml}{<GUESS_1D>} tag. 
Possible functions are the common cmath functions. 
Variables can be used in this functions, if they are later defined in the section 
\mintinline{xml}{<CONSTANTS>}.
If you change the number of grid points in one xml-file, keep in mind that the
grid points must coincide for both the ground and excited state.
In x-direction more grid points are needed.
This can be seen in the following simplified equation for the light interaction:
\begin{equation}
  \cos(kx-\Delta \omega t)
\end{equation}
To represent the cosine accurately the product $kx$ must be small enough, i.e.
many grid points in x-direction.


\subsection{Interferometer sequence}
\label{sub:interferometer_sequence}

Below is an annotated example file for an interferometer sequence shown. 
The sequence is in the Mach-Zehnder configuration.
\inputminted{xml}{bragg_ad.xml} \noindent 
Again all variables are given in dimensionless units. 
In the following a few elements will be discussed in more detail.
\begin{itemize}
  \item \mintinline{xml}{<rabi_threshold>}: Rabi oscillation are computed
    by Fourier transforming the wave function of the ground state. Let us assume
    that the momentum state of zeroth order is at point $(0,0,0)$. 
    The momentum state is spread. In order to calculate the number of particles
    in the momentum state we define an interval between $(-a,0,0)$ 
    and $(a,0,0)$ to take this width into account.
    The bounds $a$ for this interval are given by \mintinline{xml}{rabi_threshold}.
    At the moment the position of the momentum states are hard coded into the
    program. Their positions are: $(0,0,0)$, $(-\hbar k,0,0)$ and $(\hbar k,0,0)$.
    A change of momentum due to gravity for example is not considered. 
  \item \mintinline{xml}{<Amp_1>}: At the moment different amplitudes 
    are only supported for numerical diagonalisations.
\end{itemize}
There are a few flags one can use for sequence elements (also called sequence item):
\begin{minted}{xml}
  <bragg_ad dt="0.2" Nk="100" output_freq="last" pn_freq="last" 
            rabi_output_freq="each">100</bragg_ad>
\end{minted}
The duration of an interferometer phase is encased in the 
\mintinline{xml}{<bragg_ad>..</bragg_ad>} tags.
Size of time steps is given by \mintinline{xml}{dt}. \mintinline{xml}{Nk} 
is the number of iterations per output, meaning number of full kinetic steps 
instead of half steps. The flag \mintinline{xml}{output_freq} defines the number of 
files written. Possible options are:
\begin{itemize}
  \item \mintinline{xml}{none}: no file is written
  \item \mintinline{xml}{last}: after the last time step the file is written once
  \item \mintinline{xml}{each}: after \mintinline{xml}{Nk} time steps the data is written
    to file.
\end{itemize}
The particle numbers of the internal states can be printed to terminal
by the flag \mintinline{xml}{pn_freq}. The flag \mintinline{xml}{rabi_output_freq}
computes the number of particles in each momentum state. If you choose the option
\mintinline{xml}{rabi_output_freq="each"}, you get a file called xyz which contains
the Rabi oscillation between momentum states.

\subsection{Analyze files}
\label{sub:analyze_files}
Use the program \mintinline{bash}{ana_tool} to analyze wave functions.
Below is an example file showing possible options and flags.
The tool is written for Bragg beamsplitters but it should be easy to adapt
this tool to Raman beamsplitters (use wavefunctions $\Psi_1$ and $\Psi_3$ instead of 
just $\Psi_1$).
\inputminted{xml}{tools.xml} \noindent 

\section{Comparison of code and numerics}
\label{sec:comparison_code_and_numerics}

The diagonalised potential part for Bragg beamsplitters is given by equation (\ref{eq:num}). 
In the following we discuss the implementation of this equation into our 
simulator by means of an annotated source code for the potential part which is part
of the file \mintinline{bash}{atus2/source/interferometry/src/bragg.cpp}.

\begin{minted}[mathescape]{c++}
template<class T, int dim>
  void Bragg_single<T,dim>::Do_Bragg_ad()
  {
    #pragma omp parallel
    {
      // Get size of timesteps $\Delta t$
      const double dt = this->Get_dt();
      /* Current time + $0.5\Delta t$. See equation XYZ 
      */
      const double t1 = this->Get_t()+0.5*dt;

      // Pointer to m_fields which contains the wavefunctions
      // $\Psi_1$ is the groundstate and $\Psi_2$ the excited state
      fftw_complex *Psi_1 = this->m_fields[0]->Getp2In();
      fftw_complex *Psi_2 = this->m_fields[1]->Getp2In();

      fftw_complex O11, O12, O21, O22, gamma_p, gamma_m, eta;
      double re1, im1, tmp1, tmp2, V11, V22, Ep, Em, Omega;

      // x coordinate
      CPoint<dim> x;

      // For temporal pulseshapes as specified in the xml-file
      double F = this->Amplitude_at_time();

      // Loop over all grid points
      // In 2 or 3 dimension the x,y,z directions are given by
      // 
      #pragma omp for
      for ( int l=0; l<this->m_no_of_pts; l++ )
      {
        // Get x position
        x = this->m_fields[0]->Get_x(l);

        // Precalculate $|\Psi|^2$ for later use. In Psi[l][0] the first index
        // denotes the grid point and the second one the real ([0]) 
        // or imaginary part ([1])
        tmp1 = Psi_1[l][0]*Psi_1[l][0]+Psi_1[l][1]*Psi_1[l][1];
        tmp2 = Psi_2[l][0]*Psi_2[l][0]+Psi_2[l][1]*Psi_2[l][1];

        // Compute self interaction (nonlinear terms) and gravitational potential:
        // $ V_{11} = g_{11}|\Psi_1|^2 + g_{12}|\Psi_2|^2 + mgx $ 
        // $ V_{22} = g_{21}|\Psi_1|^2 + g_{22}|\Psi_2|^2 + \Delta_L + mgx $
        V11 = this->m_gs[0]*tmp1+this->m_gs[1]*tmp2+beta[0]*x[0];
        V22 = this->m_gs[2]*tmp1+this->m_gs[3]*tmp2-DeltaL[1]+beta[0]*x[0];

        // Compute light field
        // $ \Omega = F(t)E_0\cos(kx-(\Delta\omega+\nu t)t+\phi)$
        // Here is $F(t)$ the temporal shape, $\nu$ is the chirp and $\phi$ is a 
        // phase term.
        Omega = F*Amp[0]*cos(laser_k[0]*x[0]-(laser_domh[0]+chirp*t1+chirp_rate[0]*t1)*
        t1+phase[0]/2);

        // Problem: If $\Omega = 0$ division by 0.
        // Therefore compute case without light field ($\Omega=0$) separately.
        // After that do loop for next l
        if ( Omega == 0.0 )
        {
          // $\exp(-\Delta t \cdot V11)$
          sincos( -dt*V11, &im1, &re1 );
          tmp1 = Psi_1[l][0];
          // Calculate $\Psi_1 \cdot \exp(-\Delta t \cdot V11)$. 
          // This is a multiplication of two complex numbers so we get
          Psi_1[l][0] = tmp1*re1-Psi_1[l][1]*im1;
          Psi_1[l][1] = tmp1*im1+Psi_1[l][1]*re1;

          sincos( -dt*V22, &im1, &re1 );
          tmp1 = Psi_2[l][0];
          Psi_2[l][0] = tmp1*re1-Psi_2[l][1]*im1;
          Psi_2[l][1] = tmp1*im1+Psi_2[l][1]*re1;
          continue;
        }

        // Calculate $ \eta = \exp(-0.5i(\Delta kx+\phi)) $
        sincos( -0.5*(laser_dk[0]*x[0]), &im1, &re1 );
        eta[0] = re1;
        eta[1] = im1;

        // Eigenvalues $E_+$ and $E_-$
        tmp1 = sqrt((V11-V22)*(V11-V22)+4.0*Omega*Omega);
        Ep = 0.5*(V11+V22+tmp1);
        Em = 0.5*(V11+V22-tmp1);

        // Normalize $\exp(-i(\Delta t \cdot E_p))$ and save as $\gamma_p$
        sincos( -dt*Ep, &im1, &re1 );
        tmp1 = 1.0/fabs((V11-Ep)*(V11-Ep)+Omega*Omega);
        gamma_p[0] = tmp1*re1;
        gamma_p[1] = tmp1*im1;

        // Normalize $\exp(-i(\Delta t \cdot E_m))$ and save as $\gamma_m$
        sincos( -dt*Em, &im1, &re1 );
        tmp1 = 1.0/fabs((V11-Em)*(V11-Em)+Omega*Omega);
        gamma_m[0] = tmp1*re1;
        gamma_m[1] = tmp1*im1;

        // Calculate $H_{11}$ (called O11) with the $\gamma$'s (see equation XYZ)
        tmp1 = Omega*Omega;
        O11[0] = tmp1*(gamma_p[0]+gamma_m[0]);
        O11[1] = tmp1*(gamma_p[1]+gamma_m[1]);

        tmp1 = V11-Ep;
        tmp2 = V11-Em;

        //$H_{22}$
        O22[0] = tmp1*tmp1*gamma_p[0]+tmp2*tmp2*gamma_m[0];
        O22[1] = tmp1*tmp1*gamma_p[1]+tmp2*tmp2*gamma_m[1];

        //$H_{12}$ and $H_{21}$
        O12[0] = -Omega*(tmp1*gamma_p[0]+tmp2*gamma_m[0]);
        O12[1] = -Omega*(tmp1*gamma_p[1]+tmp2*gamma_m[1]);

        O21[0] = O12[0];
        O21[1] = O12[1];

        tmp1 = O12[0];
        O12[0] = eta[0]*tmp1-eta[1]*O12[1];
        O12[1] = eta[1]*tmp1+eta[0]*O12[1];

        tmp1 = O21[0];
        O21[0] = eta[0]*tmp1+eta[1]*O21[1];
        O21[1] = eta[0]*O21[1]-eta[1]*tmp1;

        // Compute potential step with $H\cdot \Psi$ (matrix * vector)
        gamma_p[0] = Psi_1[l][0];
        gamma_p[1] = Psi_1[l][1];
        gamma_m[0] = Psi_2[l][0];
        gamma_m[1] = Psi_2[l][1];

        Psi_1[l][0] = (O11[0]*gamma_p[0]-O11[1]*gamma_p[1]) + 
                      (O12[0]*gamma_m[0]-O12[1]*gamma_m[1]);
        Psi_1[l][1] = (O11[0]*gamma_p[1]+O11[1]*gamma_p[0]) + 
                      (O12[0]*gamma_m[1]+O12[1]*gamma_m[0]);

        Psi_2[l][0] = (O21[0]*gamma_p[0]-O21[1]*gamma_p[1]) + 
                      (O22[0]*gamma_m[0]-O22[1]*gamma_m[1]);
        Psi_2[l][1] = (O21[0]*gamma_p[1]+O21[1]*gamma_p[0]) + 
                      (O22[0]*gamma_m[1]+O22[1]*gamma_m[0]);
      }
    }
  }
\end{minted}


\section{Adding new sequence items}
\label{sec:adding_new_sequences}
If you want to add new sequence items, e.g. read in phase errors due to mirror
imperfections, you can do this easily by adding your source code 
to the run\_custom\_sequence function of
\mintinline{bash}{atus2/source/interferometry/src/bragg.cpp}.
An example will be added soon.


\section{Appendix}
%Further sequence flags:
%\begin{itemize}
%  \item \mintinline{xml}{no_of_chirps}
%  \item \mintinline{xml}{chirp_mode}
%\end{itemize}
Besides Bragg beam splitters you can also choose to you use one of the
two following beam splitter types:
\begin{itemize}
  \item \textbf{Double Bragg}:
        For a double Bragg beam splitter we have to use three instead of two
        wave functions. The wave functions are composed of a ground state and two
        excited states. We invoke double Bragg beam splitters by the command
        \mintinline{bash}{double_bragg}. Further information and details on the
        numerics can be found in \cite{rosskamp_modellierung_2014} (in german).
      \item \textbf{Raman}: 
        For a Raman beam splitter we need again three wave functions, this time composed
        of a ground, excited and a second ground state. An example xml-file can be
        found in \mintinline{bash}{atus2/xml/raman/raman.xml}. 
        For resonant transitions the following
        equation for the detunings must hold:
        \begin{equation}
          \Delta_{L2}+\Delta_{L3}=\Delta \omega.
        \end{equation}
        Here $\Delta_{L2}$ is the detuning to the excited state with respect to the
        ground state. $\Delta_{L3}$ is the detuning to the second ground state with
        respect to the first ground state (see xml-file). \\
        Raman beam splitters are implemented with a numerical diagonalisation of the 
        potential part. The source code for this diagonalisation is part of the file 
        \mintinline{bash}{atus2/include/CRT_Base_IF.h}. 

\end{itemize}

\bibliographystyle{plain}
\bibliography{lit}
\end{document}

